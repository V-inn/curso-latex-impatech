%
% caracteres-reservados.tex
%
% Introdução ao LaTeX no IMPA Tech
%
% 2025 Rafael Luis Beraldo
% <rafael.beraldo@impatech.org.br>
%
% Problema: o arquivo a seguir não irá compilar corretamente, pois usa símbolos
% reservados ao LaTeX. Corrija os problemas até que a compilação seja
% bem-sucedida.
%

\documentclass[12pt,a4paper,oneside]{article}
\usepackage{polyglossia}
  \setdefaultlanguage{brazil}

\begin{document}

Nossos teclados são bem limitados em termos de símbolos. Provavelmente por isso, os poucos símbolos que temos à nossa disposição foram tomando sentidos diferentes com o passar do tempo.

É muito comum, por exemplo, substituir a letra~``e'' pelo~\&. Em alguns contextos na Internet, a~\# é usada para indicar tags. Em alguns casos, \_underlines\_ são equivalentes a um texto itálico. E no \LaTeX, o símbolo~\textbackslash\ indica o começo de um comando.
\paragraph{
$$ a_1 + \frac{1}{a_2 + \frac{1}{a_3 + \cdots}} $$
}
\end{document}
