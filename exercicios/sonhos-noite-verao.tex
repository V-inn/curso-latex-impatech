%
% sonhos-noite-verao.tex
%
% Introdução ao LaTeX no IMPA Tech
%
% 2025 Rafael Luis Beraldo
% <rafael.beraldo@impatech.org.br>
%
% Problema: finalmente, você conseguiu o emprego de seus sonhos e trabalhará
% numa editora como tipógrafo, usando nada menos que LaTeX! Prove suas
% habilidades, editando o texto abaixo de acordo com as seguintes convenções:
%
% - Os nomes dos personagens devem ser tipografados em versaletes (small caps).
%   Além disso, devem ser seguidos de uma nova linha, para que fiquem acima da
%   fala do personagem. O nome do personagem e sua fala NÃO devem ser separados
%   por um novo parágrafo.
% - As ambientações devem estar em negrito.
% - Instruções de entrada e saída dos personagens devem estar em itálico.
% - Os títulos dos atos devem ser escritos com o tamanho \huge, e o tamanho dos
%   títulos das cenas deve ser \Large.
%
% (O texto abaixo foi encontrado em http://www.ebooksbrasil.org/eLibris/sonhoverao.html)
%

\documentclass[a4paper,twoside]{book}
\usepackage{fontspec}
\usepackage{polyglossia}
  \setdefaultlanguage{brazil}

\title{Sonho de uma Noite de Verão}
\author{William Shakespeare}
\date{Primeira performance: 1605}

\begin{document}
\frenchspacing

\maketitle

ATO I
Cena I

Atenas. O palácio de Teseu. Entram Teseu, Hipólita, Filóstrato e pessoas do
séquito.

TESEU: Depressa, bela Hipólita, aproxima-se a hora de nossas núpcias. Quatro
dias felizes nos trarão uma outra lua. Mas, para mim, como esta lua velha se
extingue lentamente! Ela retarda meus anelos, tal como o faz madrasta ou viúva
que retém os bens do herdeiro.

HIPÓLITA: Mergulharão depressa quatro dias na negra noite; quatro noites,
presto, farão escoar o tempo como em sonhos. E então a lua que, como arco
argênteo. no céu ora se encurva, verá a noite solene do esposório.

TESEU: Vai, Filóstrato, concita os atenienses para a festa, desperta o alegre
e buliçoso espírito da alegria, despacha para os ritos fúnebres a tristeza, que
essa pálida hóspede não vai bem em nossas pompas. (Sai Filóstrato.) De espada
em mão te fiz a corte, Hipólita; o coração te conquistei à custa de violência;
mas quero desposar-te com música de tom mais auspicioso, com pompas, com
triunfos, com festejos.

(Entram Egeu, Hérmia, Lisandro e Demétrio.)

EGEU: Salve, Teseu, nosso famoso duque!

TESEU: Bom Egeu, obrigado. Que há de novo?

EGEU: Cheio de dor, venho fazer-te queixa de minha própria filha, Hérmia
querida. Vem para cá, Demétrio. Nobre lorde, tem este homem o meu consentimento
para casar com ela. Agora avança. Lisandro. E este, meu príncipe gracioso, o
peito de Hérmia traz enfeitiçado. Sim, Lisandro, tu mesmo, com tuas rimas!
Prendas de amor com ela tu trocaste; sob a sua janela, à luz da lua,
cantaste-lhe canções com voz fingida, versos de amor fingido, e cativaste as
impressões de sua fantasia com cachos de cabelo, anéis, brinquedos, ramalhetes,
docinhos, ninharias, mensageiros de efeito decisivo nas jovens ainda brandas.
Com astúcia, à minha filha o coração furtaste, mudaste-lhe a filial obediência
em dura teimosia. Por tudo isso, meu mui gracioso duque, se ela, agora. Diante
de Vossa Graça, com Demétrio não quiser se casar, eu me reporto à antiga lei de
Atenas que confere aos pais direito de dispor dos filhos. É minha filha, posso
dispor dela. Ou a entregarei para este cavalheiro, ou para a morte, o que, sem
mais delongas, segundo nossa lei, deve ser feito.

Etc.
\end{document}
