%
% slides-LaTeX-impatech-2025.tex
%
% Introdução ao LaTeX no IMPA Tech
%
% 2025 Rafael Luis Beraldo
% <rafael.beraldo@impatech.org.br>
%
%

\documentclass[numbering=fraction,aspectratio=169]{beamer}


% -------------------------------
% Pacotes
% -------------------------------
\usepackage{fontspec}
  \setmainfont{Latin Modern Roman}
  \newfontfamily\emoji{NotoColorEmoji}[Renderer=HarfBuzz]
\usepackage{polyglossia}
  \setdefaultlanguage[variant=brazilian]{portuguese}
\usepackage{hyperref}
\usepackage{xspace}
\usepackage{graphicx}
\usepackage{multicol}
\usepackage{microtype}
\usepackage{minted}
\usepackage{tcolorbox}
\usepackage{enumerate}

% -------------------------------
% Tema e estilo
% -------------------------------
\usetheme[progressbar=frametitle]{metropolis}
\beamertemplatenavigationsymbolsempty

% Cores
\definecolor{impatechpink}{RGB}{254,137,239}
\definecolor{impatechorange}{RGB}{254,196,84}
\definecolor{impatechgreen}{RGB}{189,255,139}

% Logo na caa
\titlegraphic{\includegraphics[width=3cm]{img/impatech-logo}}

% Tema de cores e ajustes de fonte
\setbeamercolor{frametitle}{fg=black, bg=impatechorange}
\setbeamercolor{title separator}{fg=impatechorange}

\setbeamercolor{block title alerted}{fg=black, bg=impatechpink}
\setbeamercolor{block body alerted}{fg=black, bg=impatechpink!30}
\setbeamercolor{alerted text}{fg=impatechpink}
\setbeamerfont{alerted text}{series=\bfseries}

\setbeamercolor{block title example}{fg=black, bg=impatechgreen}
\setbeamercolor{block body example}{fg=black, bg=impatechgreen!30}
\setbeamercolor{footline}{fg=gray}
\setbeamercolor{progress bar}{fg=black}

% Rodapé
\setbeamertemplate{frame footer}{\insertshortauthor~(\insertshortinstitute)}


% -------------------------------
% Meus comandos
% -------------------------------
% IMPA Tech logo com pequeno espaçamento
\newcommand\IMPATech{\textsc{impa}\,Tech\xspace}
% Emoji
\newcommand\ThumbsUp{{\emoji 👍}\ }
\newcommand\ThumbsDown{{\emoji 👎}\ }
\newcommand\WarningEmoji{{\emoji ⚠️}\ }
% Itens semânticos
\newcommand{\filename}[1]{\texttt{#1}}
\newcommand{\email}[1]{\href{mailto:#1} {\texttt{\textless#1\textgreater}}}
% TODO
\newcommand{\todo}[1]{\colorbox{yellow}{#1}}
% Código
\newcommand{\code}[1]{\texttt{#1}}
\newmintinline[latexcode]{latex}{}

% -------------------------------
% URLs
% -------------------------------
% \def


% -------------------------------
% Metadados do título
% -------------------------------
\title{Introdução ao \LaTeX: dia dois}
\author[Rafael Beraldo]{Rafael Luis Beraldo\\
  \texttt{<\href{mailto:rafael.beraldo@impatech.org.br}{rafael.beraldo@impatech.org.br}>}\\
  \textbf{Introdução ao \LaTeX}}
\institute{\IMPATech}
\date{June 17, 2025}


% -------------------------------
% Documento
% -------------------------------
\begin{document}
\frenchspacing

% Título
\begin{frame}[plain]{}
  \maketitle
\end{frame}

% Conteúdo
\begin{frame}
  \frametitle{Conteúdo}
  \setbeamertemplate{section in toc}[sections numbered]
  \begin{multicols}{2}
    \tableofcontents
  \end{multicols}
\end{frame}

\input{sections/historia}
\input{sections/linguagem-marcacao}
% Exemplo de artigo %%%%%%%%%%%%%%
\section{Exemplo de artigo}

% Vejamos nosso primeiro artigo em LaTeX.
\begin{frame}
  \frametitle{Exemplo de artigo}
  \Huge
  Vejamos \filename{exemplos/artigo.tex}
\end{frame}

\input{sections/comandos}
\input{sections/espaco-branco}
\input{sections/simbolos-especiais}
\input{sections/preambulo-documento}
\input{sections/corpo-documento}
\input{sections/pacotes}
\input{sections/fontes}
%
% layouts-pagina.tex
%
% Introdução ao LaTeX no IMPA Tech
%
% 2025 Rafael Luis Beraldo
% <rafael.beraldo@impatech.org.br>
%
% Demonstra:
% - Margens com as opções onecolumn and twocolumn
% - Visualização de margens com o pacote showframe
% - O pacote fullpage e seus problemas
% - O pacote setspace e a entrelinha
% - Estilos de página
%
% O documento está vazio, pois depende da solução de sonhos-noite-verao.tex.


\end{document}

% Local Variables:
% jinx-languages: "pt_BR"
% End:
